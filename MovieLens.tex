% Options for packages loaded elsewhere
\PassOptionsToPackage{unicode}{hyperref}
\PassOptionsToPackage{hyphens}{url}
%
\documentclass[
]{article}
\usepackage{lmodern}
\usepackage{amssymb,amsmath}
\usepackage{ifxetex,ifluatex}
\ifnum 0\ifxetex 1\fi\ifluatex 1\fi=0 % if pdftex
  \usepackage[T1]{fontenc}
  \usepackage[utf8]{inputenc}
  \usepackage{textcomp} % provide euro and other symbols
\else % if luatex or xetex
  \usepackage{unicode-math}
  \defaultfontfeatures{Scale=MatchLowercase}
  \defaultfontfeatures[\rmfamily]{Ligatures=TeX,Scale=1}
\fi
% Use upquote if available, for straight quotes in verbatim environments
\IfFileExists{upquote.sty}{\usepackage{upquote}}{}
\IfFileExists{microtype.sty}{% use microtype if available
  \usepackage[]{microtype}
  \UseMicrotypeSet[protrusion]{basicmath} % disable protrusion for tt fonts
}{}
\makeatletter
\@ifundefined{KOMAClassName}{% if non-KOMA class
  \IfFileExists{parskip.sty}{%
    \usepackage{parskip}
  }{% else
    \setlength{\parindent}{0pt}
    \setlength{\parskip}{6pt plus 2pt minus 1pt}}
}{% if KOMA class
  \KOMAoptions{parskip=half}}
\makeatother
\usepackage{xcolor}
\IfFileExists{xurl.sty}{\usepackage{xurl}}{} % add URL line breaks if available
\IfFileExists{bookmark.sty}{\usepackage{bookmark}}{\usepackage{hyperref}}
\hypersetup{
  pdftitle={MovieLens},
  pdfauthor={Omkar Oka},
  hidelinks,
  pdfcreator={LaTeX via pandoc}}
\urlstyle{same} % disable monospaced font for URLs
\usepackage[margin=1in]{geometry}
\usepackage{color}
\usepackage{fancyvrb}
\newcommand{\VerbBar}{|}
\newcommand{\VERB}{\Verb[commandchars=\\\{\}]}
\DefineVerbatimEnvironment{Highlighting}{Verbatim}{commandchars=\\\{\}}
% Add ',fontsize=\small' for more characters per line
\usepackage{framed}
\definecolor{shadecolor}{RGB}{248,248,248}
\newenvironment{Shaded}{\begin{snugshade}}{\end{snugshade}}
\newcommand{\AlertTok}[1]{\textcolor[rgb]{0.94,0.16,0.16}{#1}}
\newcommand{\AnnotationTok}[1]{\textcolor[rgb]{0.56,0.35,0.01}{\textbf{\textit{#1}}}}
\newcommand{\AttributeTok}[1]{\textcolor[rgb]{0.77,0.63,0.00}{#1}}
\newcommand{\BaseNTok}[1]{\textcolor[rgb]{0.00,0.00,0.81}{#1}}
\newcommand{\BuiltInTok}[1]{#1}
\newcommand{\CharTok}[1]{\textcolor[rgb]{0.31,0.60,0.02}{#1}}
\newcommand{\CommentTok}[1]{\textcolor[rgb]{0.56,0.35,0.01}{\textit{#1}}}
\newcommand{\CommentVarTok}[1]{\textcolor[rgb]{0.56,0.35,0.01}{\textbf{\textit{#1}}}}
\newcommand{\ConstantTok}[1]{\textcolor[rgb]{0.00,0.00,0.00}{#1}}
\newcommand{\ControlFlowTok}[1]{\textcolor[rgb]{0.13,0.29,0.53}{\textbf{#1}}}
\newcommand{\DataTypeTok}[1]{\textcolor[rgb]{0.13,0.29,0.53}{#1}}
\newcommand{\DecValTok}[1]{\textcolor[rgb]{0.00,0.00,0.81}{#1}}
\newcommand{\DocumentationTok}[1]{\textcolor[rgb]{0.56,0.35,0.01}{\textbf{\textit{#1}}}}
\newcommand{\ErrorTok}[1]{\textcolor[rgb]{0.64,0.00,0.00}{\textbf{#1}}}
\newcommand{\ExtensionTok}[1]{#1}
\newcommand{\FloatTok}[1]{\textcolor[rgb]{0.00,0.00,0.81}{#1}}
\newcommand{\FunctionTok}[1]{\textcolor[rgb]{0.00,0.00,0.00}{#1}}
\newcommand{\ImportTok}[1]{#1}
\newcommand{\InformationTok}[1]{\textcolor[rgb]{0.56,0.35,0.01}{\textbf{\textit{#1}}}}
\newcommand{\KeywordTok}[1]{\textcolor[rgb]{0.13,0.29,0.53}{\textbf{#1}}}
\newcommand{\NormalTok}[1]{#1}
\newcommand{\OperatorTok}[1]{\textcolor[rgb]{0.81,0.36,0.00}{\textbf{#1}}}
\newcommand{\OtherTok}[1]{\textcolor[rgb]{0.56,0.35,0.01}{#1}}
\newcommand{\PreprocessorTok}[1]{\textcolor[rgb]{0.56,0.35,0.01}{\textit{#1}}}
\newcommand{\RegionMarkerTok}[1]{#1}
\newcommand{\SpecialCharTok}[1]{\textcolor[rgb]{0.00,0.00,0.00}{#1}}
\newcommand{\SpecialStringTok}[1]{\textcolor[rgb]{0.31,0.60,0.02}{#1}}
\newcommand{\StringTok}[1]{\textcolor[rgb]{0.31,0.60,0.02}{#1}}
\newcommand{\VariableTok}[1]{\textcolor[rgb]{0.00,0.00,0.00}{#1}}
\newcommand{\VerbatimStringTok}[1]{\textcolor[rgb]{0.31,0.60,0.02}{#1}}
\newcommand{\WarningTok}[1]{\textcolor[rgb]{0.56,0.35,0.01}{\textbf{\textit{#1}}}}
\usepackage{longtable,booktabs}
% Correct order of tables after \paragraph or \subparagraph
\usepackage{etoolbox}
\makeatletter
\patchcmd\longtable{\par}{\if@noskipsec\mbox{}\fi\par}{}{}
\makeatother
% Allow footnotes in longtable head/foot
\IfFileExists{footnotehyper.sty}{\usepackage{footnotehyper}}{\usepackage{footnote}}
\makesavenoteenv{longtable}
\usepackage{graphicx,grffile}
\makeatletter
\def\maxwidth{\ifdim\Gin@nat@width>\linewidth\linewidth\else\Gin@nat@width\fi}
\def\maxheight{\ifdim\Gin@nat@height>\textheight\textheight\else\Gin@nat@height\fi}
\makeatother
% Scale images if necessary, so that they will not overflow the page
% margins by default, and it is still possible to overwrite the defaults
% using explicit options in \includegraphics[width, height, ...]{}
\setkeys{Gin}{width=\maxwidth,height=\maxheight,keepaspectratio}
% Set default figure placement to htbp
\makeatletter
\def\fps@figure{htbp}
\makeatother
\setlength{\emergencystretch}{3em} % prevent overfull lines
\providecommand{\tightlist}{%
  \setlength{\itemsep}{0pt}\setlength{\parskip}{0pt}}
\setcounter{secnumdepth}{5}

\title{\textbf{MovieLens}}
\author{\textbf{Omkar Oka}}
\date{\textbf{6/15/2020}}

\begin{document}
\maketitle

{
\setcounter{tocdepth}{2}
\tableofcontents
}
\newpage

\hypertarget{executive-summary}{%
\section{\texorpdfstring{\textbf{Executive
summary:}}{Executive summary:}}\label{executive-summary}}

Recommendation is the act of generating a suitable suggestion based on a
user's shopping pattern and their reviews for certain products , music
or movies either on a shopping or a streaming platform. Making the right
recommendation for the next product, music or movie increases user
retention and satisfaction, leading to sales and profit growth.

The most famous project that tried to achieve that goal was the
\textbf{The Netflix Prize} (October 2006). This project was an open
competition to predict user ratings for films, based on previous ratings
without any other information about the users or films. The goal was to
make the company's recommendation engine 10\% more accurate.

In this project we try to generate a model that can predict the rating
that a user will give to a movie based on their preference and previous
interactions, in terms of genres and the overall effects of the movie.
We will be using the MovieLens data set pre-processed by the GroupLens
research lab in the University of Minnesota.

Further we will conduct an exploratory analysis of the data and based on
some characteristics of the data we can decide the features that will be
used to create a Machine Learning algorithm. This document explores step
by step approach, process, techniques and methods that were used to
handle the data and to create the predictive model.

The final section shows the results of the previous process and then,
the conclusion of the project with a future road map. Our goal is to
achieve an RMSE of less than 0.86490.

\hypertarget{evaluation-model}{%
\subsection{\texorpdfstring{\textbf{Evaluation
Model:}}{Evaluation Model:}}\label{evaluation-model}}

The evaluation of machine learning algorithm consists comparing the
predicted value with the actual outcome. The loss function measures the
difference between both values. We will be using the Root Mean Squared
Error method for evaluating the performance of our model.

The Root Mean Squared Error, RMSE, is the square root of the MSE. It is
the typical metric to evaluate recommendation systems, and is defined by
the formula:

\[RMSE=\sqrt{\frac{1}{N}\sum_{u,i}(\hat{y}_{u,i}-y_{u,i})^2}\]

where N is the number of ratings, \(y_{u,i}\) is the rating of movie i
by user u and \(\hat{y}_{u,i}\) is the prediction of movie i by user u.

RMSE penalizes large deviations from the mean and is appropriate in
cases that small errors are not relevant. Contrary to other methods, the
error has the same unit as the measurement.

\hypertarget{data-preparation}{%
\subsection{\texorpdfstring{\textbf{Data
Preparation:}}{Data Preparation:}}\label{data-preparation}}

In this section we download and prepare the dataset to be used in the
analysis. We split the data set in two parts, the training set called
edx and the evaluation set called validation with 90\% and 10\% of the
original dataset respectively.

Then, we split the edx set in two parts, the train set and test set with
90\% and 10\% of edx set respectively. The model is created and trained
in the train set and tested in the test set until the RMSE target is
achieved, then finally we train the model again in the entire edx set
and validate in the validation set. The name of this method is
cross-validation.

\newpage

Extracting data publicly available from GroupLens:

\begin{Shaded}
\begin{Highlighting}[]
 \CommentTok{#Create test and validation sets}
 \CommentTok{#Create edx set, validation set, and submission file}
\KeywordTok{options}\NormalTok{(}\DataTypeTok{warn=}\OperatorTok{-}\DecValTok{1}\NormalTok{) }\CommentTok{#Suppress Warnings}
\ControlFlowTok{if}\NormalTok{(}\OperatorTok{!}\KeywordTok{require}\NormalTok{(tidyverse)) }\KeywordTok{install.packages}\NormalTok{(}\StringTok{"tidyverse"}\NormalTok{, }\DataTypeTok{repos =} \StringTok{"http://cran.us.r-project.org"}\NormalTok{)}
\ControlFlowTok{if}\NormalTok{(}\OperatorTok{!}\KeywordTok{require}\NormalTok{(caret)) }\KeywordTok{install.packages}\NormalTok{(}\StringTok{"caret"}\NormalTok{, }\DataTypeTok{repos =} \StringTok{"http://cran.us.r-project.org"}\NormalTok{)}
\ControlFlowTok{if}\NormalTok{(}\OperatorTok{!}\KeywordTok{require}\NormalTok{(data.table)) }\KeywordTok{install.packages}\NormalTok{(}\StringTok{"data.table"}\NormalTok{, }\DataTypeTok{repos =} \StringTok{"http://cran.us.r-project.org"}\NormalTok{)}
\ControlFlowTok{if}\NormalTok{(}\OperatorTok{!}\KeywordTok{require}\NormalTok{(anytime)) }\KeywordTok{install.packages}\NormalTok{(}\StringTok{"anytime"}\NormalTok{, }\DataTypeTok{repos =} \StringTok{"http://cran.us.r-project.org"}\NormalTok{)}
\ControlFlowTok{if}\NormalTok{(}\OperatorTok{!}\KeywordTok{require}\NormalTok{(lubridate)) }\KeywordTok{install.packages}\NormalTok{(}\StringTok{"lubridate"}\NormalTok{, }\DataTypeTok{repos =} \StringTok{"http://cran.us.r-project.org"}\NormalTok{)}
\ControlFlowTok{if}\NormalTok{(}\OperatorTok{!}\KeywordTok{require}\NormalTok{(corrr)) }\KeywordTok{install.packages}\NormalTok{(}\StringTok{"corrr"}\NormalTok{, }\DataTypeTok{repos =} \StringTok{"http://cran.us.r-project.org"}\NormalTok{)}
\CommentTok{#Loading Libraries:}
\KeywordTok{library}\NormalTok{(}\StringTok{"stringr"}\NormalTok{)}
\KeywordTok{library}\NormalTok{(}\StringTok{"tidyverse"}\NormalTok{)}
\KeywordTok{library}\NormalTok{(}\StringTok{"caret"}\NormalTok{)}
\KeywordTok{library}\NormalTok{(}\StringTok{"anytime"}\NormalTok{)}
\KeywordTok{library}\NormalTok{(}\StringTok{"lubridate"}\NormalTok{)}
\KeywordTok{library}\NormalTok{(}\StringTok{"corrr"}\NormalTok{)}

\CommentTok{# MovieLens 10M dataset:}
 \CommentTok{# https://grouplens.org/datasets/movielens/10m/}
 \CommentTok{# http://files.grouplens.org/datasets/movielens/ml-10m.zip}

\NormalTok{dl <-}\StringTok{ }\KeywordTok{tempfile}\NormalTok{()}
 \KeywordTok{download.file}\NormalTok{(}\StringTok{"http://files.grouplens.org/datasets/movielens/ml-10m.zip"}\NormalTok{, dl)}

\NormalTok{ratings <-}\StringTok{ }\KeywordTok{fread}\NormalTok{(}\DataTypeTok{text =} \KeywordTok{gsub}\NormalTok{(}\StringTok{"::"}\NormalTok{, }\StringTok{"}\CharTok{\textbackslash{}t}\StringTok{"}\NormalTok{, }\KeywordTok{readLines}\NormalTok{(}\KeywordTok{unzip}\NormalTok{(dl, }\StringTok{"ml-10M100K/ratings.dat"}\NormalTok{))),}
                 \DataTypeTok{col.names =} \KeywordTok{c}\NormalTok{(}\StringTok{"userId"}\NormalTok{, }\StringTok{"movieId"}\NormalTok{, }\StringTok{"rating"}\NormalTok{, }\StringTok{"timestamp"}\NormalTok{))}

\NormalTok{movies <-}\StringTok{ }\KeywordTok{str_split_fixed}\NormalTok{(}\KeywordTok{readLines}\NormalTok{(}\KeywordTok{unzip}\NormalTok{(dl, }\StringTok{"ml-10M100K/movies.dat"}\NormalTok{)), }\StringTok{"}\CharTok{\textbackslash{}\textbackslash{}}\StringTok{::"}\NormalTok{, }\DecValTok{3}\NormalTok{)}
 \KeywordTok{colnames}\NormalTok{(movies) <-}\StringTok{ }\KeywordTok{c}\NormalTok{(}\StringTok{"movieId"}\NormalTok{, }\StringTok{"title"}\NormalTok{, }\StringTok{"genres"}\NormalTok{)}
                                        
\NormalTok{movies <-}\StringTok{ }\KeywordTok{as.data.frame}\NormalTok{(movies) }\OperatorTok\StringTok{ }\KeywordTok{mutate}\NormalTok{(}\DataTypeTok{movieId =} \KeywordTok{as.numeric}\NormalTok{(movieId), }\DataTypeTok{title =} \KeywordTok{as.character}\NormalTok{(title), }\DataTypeTok{genres =} \KeywordTok{as.character}\NormalTok{(genres))}
\NormalTok{movielens <-}\StringTok{ }\KeywordTok{left_join}\NormalTok{(ratings, movies, }\DataTypeTok{by =} \StringTok{"movieId"}\NormalTok{)}
 
\CommentTok{# Validation set will be 10% of MovieLens data}
\KeywordTok{set.seed}\NormalTok{(}\DecValTok{1}\NormalTok{, }\DataTypeTok{sample.kind=}\StringTok{"Rounding"}\NormalTok{)}
\NormalTok{test_index <-}\StringTok{ }\KeywordTok{createDataPartition}\NormalTok{(}\DataTypeTok{y =}\NormalTok{ movielens}\OperatorTok{$}\NormalTok{rating, }\DataTypeTok{times =} \DecValTok{1}\NormalTok{, }\DataTypeTok{p =} \FloatTok{0.1}\NormalTok{, }\DataTypeTok{list =} \OtherTok{FALSE}\NormalTok{)}
\NormalTok{edx <-}\StringTok{ }\NormalTok{movielens[}\OperatorTok{-}\NormalTok{test_index,]}
\NormalTok{temp <-}\StringTok{ }\NormalTok{movielens[test_index,]}
\CommentTok{# Make sure userId and movieId in validation set are also in edx set}
\NormalTok{validation <-}\StringTok{ }\NormalTok{temp }\OperatorTok\StringTok{ }\KeywordTok{semi_join}\NormalTok{(edx, }\DataTypeTok{by =} \StringTok{"movieId"}\NormalTok{) }\OperatorTok\StringTok{ }\KeywordTok{semi_join}\NormalTok{(edx, }\DataTypeTok{by =} \StringTok{"userId"}\NormalTok{)}
\CommentTok{# Add rows removed from validation set back into edx set}
\NormalTok{removed <-}\StringTok{ }\KeywordTok{anti_join}\NormalTok{(temp, validation)}
\NormalTok{edx <-}\StringTok{ }\KeywordTok{rbind}\NormalTok{(edx, removed)}
\KeywordTok{rm}\NormalTok{(dl, ratings, movies, test_index, temp, movielens, removed)}
\end{Highlighting}
\end{Shaded}

\newpage

\hypertarget{dataset}{%
\subsection{\texorpdfstring{\textbf{DataSet:}}{DataSet:}}\label{dataset}}

\begin{Shaded}
\begin{Highlighting}[]
\KeywordTok{str}\NormalTok{(edx)}
\end{Highlighting}
\end{Shaded}

\begin{verbatim}
## Classes 'data.table' and 'data.frame':   9000055 obs. of  6 variables:
##  $ userId   : int  1 1 1 1 1 1 1 1 1 1 ...
##  $ movieId  : num  122 185 292 316 329 355 356 362 364 370 ...
##  $ rating   : num  5 5 5 5 5 5 5 5 5 5 ...
##  $ timestamp: int  838985046 838983525 838983421 838983392 838983392 838984474 838983653 838984885 838983707 838984596 ...
##  $ title    : chr  "Boomerang (1992)" "Net, The (1995)" "Outbreak (1995)" "Stargate (1994)" ...
##  $ genres   : chr  "Comedy|Romance" "Action|Crime|Thriller" "Action|Drama|Sci-Fi|Thriller" "Action|Adventure|Sci-Fi" ...
##  - attr(*, ".internal.selfref")=<externalptr>
\end{verbatim}

The data set contains 9000055 observations of 6 variables.

\begin{Shaded}
\begin{Highlighting}[]
\KeywordTok{head}\NormalTok{(edx)}
\end{Highlighting}
\end{Shaded}

\begin{verbatim}
##    userId movieId rating timestamp                         title
## 1:      1     122      5 838985046              Boomerang (1992)
## 2:      1     185      5 838983525               Net, The (1995)
## 3:      1     292      5 838983421               Outbreak (1995)
## 4:      1     316      5 838983392               Stargate (1994)
## 5:      1     329      5 838983392 Star Trek: Generations (1994)
## 6:      1     355      5 838984474       Flintstones, The (1994)
##                           genres
## 1:                Comedy|Romance
## 2:         Action|Crime|Thriller
## 3:  Action|Drama|Sci-Fi|Thriller
## 4:       Action|Adventure|Sci-Fi
## 5: Action|Adventure|Drama|Sci-Fi
## 6:       Children|Comedy|Fantasy
\end{verbatim}

The Data Description is as follows:\\
- \texttt{userId}: Unique identification number given to each user.
\texttt{numeric} variable\\
- \texttt{movieId}: Unique identification number given to each movie.
\texttt{numeric} variable.\\
- \texttt{timestamp}: Code that contains date and time in what the
rating was given by the user to the specific movie. \texttt{integer}
variable.\\
- \texttt{title}: Title of the movie. \texttt{character} variable.\\
- \texttt{genres}: Motion-picture category associated to the film.
\texttt{character} variable.\\
- \texttt{rating}: Rating given by the user to the movie. From 0 to 5
\emph{stars} in steps of 0.5. \texttt{numeric} variable.\\
\newpage

\hypertarget{analysis-section}{%
\section{\texorpdfstring{\textbf{Analysis
Section:}}{Analysis Section:}}\label{analysis-section}}

We will be performing necessary transformations, data preparation and
exploratory data analysis as part of this section.

\hypertarget{data-format}{%
\subsection{\texorpdfstring{\textbf{Data
format:}}{Data format:}}\label{data-format}}

The \texttt{userId} and \texttt{movieId} variables are \texttt{numeric}
columns in the original data set. However it does not make sense. The
\texttt{userId}=2 is not two times the \texttt{userId}=1 and the same
effect happens with the \texttt{movieId} variable. These characteristics
are just \emph{labels}, therefore they will be converted to
\texttt{factor} type to be useful.

Both \texttt{movieId} and \texttt{title} variables give us the same
exact information. They are the \textbf{unique identification code} to
each film. We could say that these pair of variables have 100\%
correlation! Only the \texttt{movieId} column will be used and will be
used as a \texttt{factor}. It optimizes the memory (RAM) usage.

The \texttt{timestamp} variable is converted to \texttt{POSIXct} type,
to be handle correctly as a \texttt{date} vector. The year is extracted
to the \texttt{year} column and the \texttt{timestamp} column is
dropped.

\begin{verbatim}
##    userId movieId rating           timestamp                  title year
## 1:      1     122      5 1996-08-02 16:54:06              Boomerang 1992
## 2:      1     185      5 1996-08-02 16:28:45               Net, The 1995
## 3:      1     292      5 1996-08-02 16:27:01               Outbreak 1995
## 4:      1     316      5 1996-08-02 16:26:32               Stargate 1994
## 5:      1     329      5 1996-08-02 16:26:32 Star Trek: Generations 1994
## 6:      1     355      5 1996-08-02 16:44:34       Flintstones, The 1994
##                           genres year_rate
## 1:                Comedy|Romance      1996
## 2:         Action|Crime|Thriller      1996
## 3:  Action|Drama|Sci-Fi|Thriller      1996
## 4:       Action|Adventure|Sci-Fi      1996
## 5: Action|Adventure|Drama|Sci-Fi      1996
## 6:       Children|Comedy|Fantasy      1996
\end{verbatim}

\hypertarget{exploratory-analysis}{%
\subsection{\texorpdfstring{\textbf{Exploratory
Analysis:}}{Exploratory Analysis:}}\label{exploratory-analysis}}

The target is to create a model capable of predicting the variable
\texttt{rating}.

\begin{verbatim}
##    Min. 1st Qu.  Median    Mean 3rd Qu.    Max. 
##   0.500   3.000   4.000   3.512   4.000   5.000
\end{verbatim}

As we can see from the distribution of the ratings the minimum rating is
0.5 whereas the first Quantile starts at 3.0 and the mean is 3.5 with
the third Quantile at 4.0 which means that most of our ratings are in
between 3 and 4 up to 4.5

Users have the option to choose a rating value from 0.5 to 5.0, totaling
10 possible values. This is an unusual scale, so all the movies get a
rounded value rating, as shown in the chart below.

We can convert these ratings into factors and design the Machine
Learning Algorithm as a classification model. But, since we have a
target RMSE that we need to beat, its better to create a regression
model since RMSE is not usually performed on a classification model.

\newpage

\textbf{Overall Distribution of ratings:}

\begin{longtable}[]{@{}rrl@{}}
\caption{Rating\_Distribution}\tabularnewline
\toprule
rating & n & Percentage\tabularnewline
\midrule
\endfirsthead
\toprule
rating & n & Percentage\tabularnewline
\midrule
\endhead
4.0 & 2588430 & 28.76\%\tabularnewline
3.0 & 2121240 & 23.57\%\tabularnewline
5.0 & 1390114 & 15.45\%\tabularnewline
3.5 & 791624 & 8.8\%\tabularnewline
2.0 & 711422 & 7.9\%\tabularnewline
4.5 & 526736 & 5.85\%\tabularnewline
1.0 & 345679 & 3.84\%\tabularnewline
2.5 & 333010 & 3.7\%\tabularnewline
1.5 & 106426 & 1.18\%\tabularnewline
0.5 & 85374 & 0.95\%\tabularnewline
\bottomrule
\end{longtable}

\begin{center}\includegraphics{MovieLens_files/figure-latex/plot_rating-1} \end{center}

We can see that our \texttt{rating} variable has a left-skewed
distribution. It's interesting that there are more \emph{good} ratings
than \emph{bad} ratings. That could be explained by the fact that people
want to recommend a film when they like it, but we could just assume
that, since there is no data available with this data set to prove this
theory.\\
\newpage

\textbf{Important Data Players:}

\begin{longtable}[]{@{}rrr@{}}
\caption{Unique\_Numbers}\tabularnewline
\toprule
Unique\_Users & Unique\_Movies & Unique\_Genres\tabularnewline
\midrule
\endfirsthead
\toprule
Unique\_Users & Unique\_Movies & Unique\_Genres\tabularnewline
\midrule
\endhead
69878 & 10677 & 797\tabularnewline
\bottomrule
\end{longtable}

We observe that there are 69878 unique users given ratings to 10677
different films. It's good to remember that the \emph{unique genres}
were counted as \texttt{factor} with no previous separation so,
\texttt{Drama} and \texttt{Comedy\textbar{}Drama} are counted as 2
different genres.

\textbf{Rates by Release Year:}\\
\includegraphics{MovieLens_files/figure-latex/year_release-1.pdf}

The frequency of ratings by \emph{release year} of the films has a clear
left skewed distribution. The most of those year are between 1990 and
2009. Primary reason for this could be that since there is a clear age
group that actively assign ratings and their preference is usually
influenced by the era of the movies.

\textbf{Average Movie Rating Distribution:}\\
\includegraphics{MovieLens_files/figure-latex/average_rating-1.pdf}

This is also a slightly left skewed distribution, since the average
rating for most of the movies ranges from 2.5 to 4.0 stars.\\
\newpage \textbf{Genres and Ratings:}

\begin{longtable}[]{@{}lrrrl@{}}
\caption{Genres\_vs\_Ratings}\tabularnewline
\toprule
genres & mean & median & n & Percentage\tabularnewline
\midrule
\endfirsthead
\toprule
genres & mean & median & n & Percentage\tabularnewline
\midrule
\endhead
Animation\textbar IMAX\textbar Sci-Fi & 4.714286 & 5.0 & 7 &
0\%\tabularnewline
Drama\textbar Film-Noir\textbar Romance & 4.304115 & 4.5 & 2989 &
0.03\%\tabularnewline
Action\textbar Crime\textbar Drama\textbar IMAX & 4.297068 & 4.5 & 2353
& 0.03\%\tabularnewline
Animation\textbar Children\textbar Comedy\textbar Crime & 4.275429 & 4.5
& 7167 & 0.08\%\tabularnewline
Film-Noir\textbar Mystery & 4.239479 & 4.0 & 5988 &
0.07\%\tabularnewline
Crime\textbar Film-Noir\textbar Mystery & 4.216803 & 4.0 & 4029 &
0.04\%\tabularnewline
Film-Noir\textbar Romance\textbar Thriller & 4.216470 & 4.0 & 2453 &
0.03\%\tabularnewline
Crime\textbar Film-Noir\textbar Thriller & 4.210157 & 4.0 & 4844 &
0.05\%\tabularnewline
Crime\textbar Mystery\textbar Thriller & 4.198981 & 4.0 & 26892 &
0.3\%\tabularnewline
Action\textbar Adventure\textbar Comedy\textbar Fantasy\textbar Romance
& 4.195557 & 4.0 & 14809 & 0.16\%\tabularnewline
Crime\textbar Thriller\textbar War & 4.171273 & 4.0 & 4595 &
0.05\%\tabularnewline
Film-Noir\textbar Mystery\textbar Thriller & 4.164298 & 4.0 & 4011 &
0.04\%\tabularnewline
Adventure\textbar Drama\textbar Film-Noir\textbar Sci-Fi\textbar Thriller
& 4.148384 & 4.0 & 13957 & 0.16\%\tabularnewline
Adventure\textbar Animation\textbar Children\textbar Comedy\textbar Sci-Fi
& 4.147917 & 4.0 & 3529 & 0.04\%\tabularnewline
Adventure\textbar Comedy\textbar Romance\textbar War & 4.127992 & 4.0 &
5223 & 0.06\%\tabularnewline
\bottomrule
\end{longtable}

The top 10 \textbf{genres} listed above have the highest \textbf{mean}.
The first place ``Animation\textbar IMAX\textbar Sci-Fi'' is clearly not
significant, because of the only 7 observations. This \emph{genre} will
be eliminated below. \textbf{Drama} is present in the 2nd and 3rd place.

\begin{longtable}[]{@{}lrrr@{}}
\caption{Top\_15\_Genres}\tabularnewline
\toprule
genres & mean & median & n\tabularnewline
\midrule
\endfirsthead
\toprule
genres & mean & median & n\tabularnewline
\midrule
\endhead
Drama & 4.304115 & 4.5 & 2989\tabularnewline
Film-Noir & 4.304115 & 4.5 & 2989\tabularnewline
Romance & 4.304115 & 4.5 & 2989\tabularnewline
Action & 4.297068 & 4.5 & 2353\tabularnewline
Crime & 4.297068 & 4.5 & 2353\tabularnewline
Drama & 4.297068 & 4.5 & 2353\tabularnewline
IMAX & 4.297068 & 4.5 & 2353\tabularnewline
Animation & 4.275429 & 4.5 & 7167\tabularnewline
Children & 4.275429 & 4.5 & 7167\tabularnewline
Comedy & 4.275429 & 4.5 & 7167\tabularnewline
Crime & 4.275429 & 4.5 & 7167\tabularnewline
Film-Noir & 4.239479 & 4.0 & 5988\tabularnewline
Mystery & 4.239479 & 4.0 & 5988\tabularnewline
Crime & 4.216803 & 4.0 & 4029\tabularnewline
Film-Noir & 4.216803 & 4.0 & 4029\tabularnewline
\bottomrule
\end{longtable}

The genres associated with the highest mean are ``Drama'', ``Film-Noir''
and ``Romance''. The \textbf{difference} with the \emph{second section}
entry numbers from 4 till 7 in that ranking is almost zero.\\
\newpage \textbf{Ratings over the Years:}\\
\includegraphics{MovieLens_files/figure-latex/time_effect_ratings-1.pdf}

Here are a few of the movies with good overall average ratings rated by
a significant number of users over the years. There seems to be a
definite time effect on the ratings which could be related to the Actors
or the similarity of the movies with the current ones playing in a
theater at the time. The time effect, actor influence or any other
effect are out of the scope of this document but it still shows that
there is a premise to explore further on these aspects.\\
\newpage \#\# \textbf{Correlation Matrix:}

\begin{longtable}[]{@{}lrrrr@{}}
\caption{Feature\_VS\_Rating\_Correlation}\tabularnewline
\toprule
rowname & userId & movieId & rating & genres\tabularnewline
\midrule
\endfirsthead
\toprule
rowname & userId & movieId & rating & genres\tabularnewline
\midrule
\endhead
userId & NA & 0.0060875 & 0.0023208 & -0.0030424\tabularnewline
movieId & 0.0060875 & NA & -0.0282943 & -0.0014997\tabularnewline
rating & 0.0023208 & -0.0282943 & NA & 0.0577063\tabularnewline
genres & -0.0030424 & -0.0014997 & 0.0577063 & NA\tabularnewline
\bottomrule
\end{longtable}

Most of the Machine Learning algorithms prefer a comparable correlation
within the features and between the features and the target(rating in
this case). Since, as we can see the \texttt{UserId}, \texttt{MovieId}
and \texttt{Genres} have a weak correlation with \texttt{Rating} any
kind of supervised Machine Learning algorithm will not have much effect
on the rating predictions and therefore will not be able to provide
useful recommendations.

Hence we will be sticking to a simple form of Linear Regression for
Predicting ratings, which is covered in the next section.\\
\newpage

\hypertarget{the-model}{%
\section{\texorpdfstring{\textbf{The
Model:}}{The Model:}}\label{the-model}}

Creating a recommendation system involves the identification of the most
important features that helps to predict the rating any given user will
give to any movie. We start building a very simple model, which is just
the mean of the observed values. Then, the user and movie effects are
included in the linear model, improving the RMSE. Finally, the user and
movie effects receive regularization parameter that penalizes samples
with few ratings.

\hypertarget{train-and-test-set}{%
\subsection{\texorpdfstring{\textbf{Train and Test
set:}}{Train and Test set:}}\label{train-and-test-set}}

First of all, we create \texttt{train} and \texttt{test} sets.

\begin{Shaded}
\begin{Highlighting}[]
\NormalTok{edx <-}\StringTok{ }\NormalTok{edx }\OperatorTok\StringTok{ }\KeywordTok{select}\NormalTok{(userId, movieId, rating)}
\KeywordTok{set.seed}\NormalTok{(}\DecValTok{1}\NormalTok{, }\DataTypeTok{sample.kind=}\StringTok{"Rounding"}\NormalTok{)}
\CommentTok{#set.seed(1) #for R version 3.5 and below}
\NormalTok{test_index <-}\StringTok{ }\KeywordTok{createDataPartition}\NormalTok{(edx}\OperatorTok{$}\NormalTok{rating, }\DataTypeTok{times =} \DecValTok{1}\NormalTok{, }\DataTypeTok{p =} \FloatTok{.1}\NormalTok{, }\DataTypeTok{list =} \OtherTok{FALSE}\NormalTok{)}
\CommentTok{# Create the index}

\NormalTok{train <-}\StringTok{ }\NormalTok{edx[}\OperatorTok{-}\NormalTok{test_index, ] }\CommentTok{# Create Train set}
\NormalTok{temp <-}\StringTok{ }\NormalTok{edx[test_index, ] }\CommentTok{# Create Test set}

\CommentTok{# Make sure userId and movieId in test set are also in train set}
\NormalTok{test <-}\StringTok{ }\NormalTok{temp }\OperatorTok\StringTok{ }
\StringTok{  }\KeywordTok{semi_join}\NormalTok{(train, }\DataTypeTok{by =} \StringTok{"movieId"}\NormalTok{) }\OperatorTok
\StringTok{  }\KeywordTok{semi_join}\NormalTok{(train, }\DataTypeTok{by =} \StringTok{"userId"}\NormalTok{)}

\CommentTok{# Add rows removed from test set back into train set}
\NormalTok{removed <-}\StringTok{ }\KeywordTok{anti_join}\NormalTok{(temp, test)}
\NormalTok{train <-}\StringTok{ }\KeywordTok{rbind}\NormalTok{(train, removed)}

\KeywordTok{rm}\NormalTok{(test_index, temp, removed)}
\end{Highlighting}
\end{Shaded}

\begin{Shaded}
\begin{Highlighting}[]
\KeywordTok{dim}\NormalTok{(train)}
\end{Highlighting}
\end{Shaded}

\begin{verbatim}
## [1] 8100065       3
\end{verbatim}

\begin{Shaded}
\begin{Highlighting}[]
\KeywordTok{dim}\NormalTok{(test)}
\end{Highlighting}
\end{Shaded}

\begin{verbatim}
## [1] 899990      3
\end{verbatim}

\hypertarget{baseline-model}{%
\subsection{\texorpdfstring{\textbf{Baseline
model:}}{Baseline model:}}\label{baseline-model}}

The most basic model is generated when we consider the most common
rating from the \emph{train} set to be predicted into the test set. This
is the \textbf{baseline} model.

\begin{verbatim}
## [1] 1.060054
\end{verbatim}

Now, we have the baseline RMSE to be \emph{beaten} by our model.

\begin{longtable}[]{@{}lr@{}}
\caption{RMSEs}\tabularnewline
\toprule
Method & RMSE\tabularnewline
\midrule
\endfirsthead
\toprule
Method & RMSE\tabularnewline
\midrule
\endhead
Baseline & 1.060054\tabularnewline
\bottomrule
\end{longtable}

We can observe that the RMSE of the most basic model is 1.0600537. It's
bigger than 1! In this context, this is a very bad model.

\hypertarget{user-and-movie-effect-model}{%
\subsection{\texorpdfstring{\textbf{User and Movie effect
Model:}}{User and Movie effect Model:}}\label{user-and-movie-effect-model}}

The next step is to improve our model and get a better RMSE by
introducing \emph{user effect} \((u_i)\) and the \emph{movie effect}
\((m_i)\) as predictors. As seen in our exploratory analysis above,
there is a definite impact of \emph{user effect} with certain users
assigning higher than usual or lower than usual ratings based on the
their preference which usually varies from \texttt{0.5} to \texttt{0.7}
in either direction.\\
\emph{Movie effect} implies that there are certain movies that generally
get rated higher than average based on their popularity. The movie
effect can be calculated as the mean of the difference between the
observed rating y and the mean \(\mu\).

\textbf{Movie Effect:} \[m_i=\frac{1}{N}\sum_{i=1}^{N}(y_i-\hat{\mu})\]
\textbf{User Effect:}
\[u_i=\frac{1}{N}\sum_{i=1}^{N}(y_{u,i}-\hat{b}_i-\hat{\mu})\]

Therefore, we are generating the next model to predict \texttt{rating}
\((\hat{y}_i)\): \[\hat{y}_i=u_i+m_i+\varepsilon\] While calculating
ratings as continuous values we can assume that there will be some
predicted ratings that might be less than zero or greater than 5. We
will be applying a balancing condition to counteract these values. We
can call this the error\((\varepsilon)\) effect and it has been adjusted
in the code already.

\textbf{RMSE obtained by calculating the \emph{user effect} and
\emph{movie effect}:}

\begin{verbatim}
## [1] 0.8244714
\end{verbatim}

\begin{longtable}[]{@{}lr@{}}
\caption{RMSE}\tabularnewline
\toprule
Method & RMSE\tabularnewline
\midrule
\endfirsthead
\toprule
Method & RMSE\tabularnewline
\midrule
\endhead
Baseline & 1.0600537\tabularnewline
User \& Movie Effect & 0.8244714\tabularnewline
\bottomrule
\end{longtable}

We've obtained a better RMSE. Now it is time to test the predictions on
Validation data.\\
\newpage

\hypertarget{user-and-movie-effect-model-on-validation-data}{%
\subsection{\texorpdfstring{\textbf{User and Movie effect Model on
\emph{validation}
data:}}{User and Movie effect Model on validation data:}}\label{user-and-movie-effect-model-on-validation-data}}

First of all, the \emph{validation} data set needs to be handled the
same was as the \emph{train} data set was handled.

\begin{Shaded}
\begin{Highlighting}[]
\NormalTok{validation <-}\StringTok{ }\NormalTok{validation }\OperatorTok\StringTok{ }\KeywordTok{select}\NormalTok{(userId, movieId, rating)}
\NormalTok{validation}\OperatorTok{$}\NormalTok{userId <-}\StringTok{ }\KeywordTok{as.factor}\NormalTok{(validation}\OperatorTok{$}\NormalTok{userId)}
\NormalTok{validation}\OperatorTok{$}\NormalTok{movieId <-}\StringTok{ }\KeywordTok{as.factor}\NormalTok{(validation}\OperatorTok{$}\NormalTok{movieId)}
\NormalTok{validation <-}\StringTok{ }\NormalTok{validation[}\KeywordTok{complete.cases}\NormalTok{(validation), ]}
\end{Highlighting}
\end{Shaded}

Now, we are ready to predict on the validation data set.

\begin{Shaded}
\begin{Highlighting}[]
\NormalTok{mu <-}\StringTok{ }\KeywordTok{mean}\NormalTok{(edx}\OperatorTok{$}\NormalTok{rating)}
\NormalTok{movie_avgs <-}\StringTok{ }\NormalTok{edx }\OperatorTok
\StringTok{  }\KeywordTok{group_by}\NormalTok{(movieId) }\OperatorTok
\StringTok{  }\KeywordTok{summarize}\NormalTok{(}\DataTypeTok{m_i =} \KeywordTok{mean}\NormalTok{(rating }\OperatorTok{-}\StringTok{ }\NormalTok{mu))}
\NormalTok{user_avgs <-}\StringTok{ }\NormalTok{edx }\OperatorTok
\StringTok{  }\KeywordTok{left_join}\NormalTok{(movie_avgs, }\DataTypeTok{by =} \StringTok{"movieId"}\NormalTok{) }\OperatorTok
\StringTok{  }\KeywordTok{group_by}\NormalTok{(userId) }\OperatorTok
\StringTok{  }\KeywordTok{summarize}\NormalTok{(}\DataTypeTok{u_i =} \KeywordTok{mean}\NormalTok{(rating }\OperatorTok{-}\StringTok{ }\NormalTok{mu }\OperatorTok{-}\StringTok{ }\NormalTok{m_i))}
\NormalTok{predicted_ratings <-}\StringTok{ }\NormalTok{validation }\OperatorTok
\StringTok{  }\KeywordTok{left_join}\NormalTok{(movie_avgs, }\DataTypeTok{by =} \StringTok{"movieId"}\NormalTok{) }\OperatorTok
\StringTok{  }\KeywordTok{left_join}\NormalTok{(user_avgs, }\DataTypeTok{by =} \StringTok{"userId"}\NormalTok{) }\OperatorTok
\StringTok{  }\KeywordTok{mutate}\NormalTok{(}\DataTypeTok{pred =}\NormalTok{ mu }\OperatorTok{+}\StringTok{ }\NormalTok{m_i }\OperatorTok{+}\StringTok{ }\NormalTok{u_i) }\OperatorTok\StringTok{ }\KeywordTok{mutate}\NormalTok{(}\DataTypeTok{pred =} \KeywordTok{ifelse}\NormalTok{(pred }\OperatorTok{<}\StringTok{ }\DecValTok{0}\NormalTok{, }\DecValTok{0}\NormalTok{, }\KeywordTok{ifelse}\NormalTok{(pred }\OperatorTok{>}\StringTok{ }\DecValTok{5}\NormalTok{, }\DecValTok{5}\NormalTok{, pred))) }\OperatorTok\StringTok{ }\NormalTok{.}\OperatorTok{$}\NormalTok{pred}

\NormalTok{val_RMSE <-}\StringTok{ }\KeywordTok{RMSE}\NormalTok{(predicted_ratings, validation}\OperatorTok{$}\NormalTok{rating, }\DataTypeTok{na.rm =}\NormalTok{ T)}
\end{Highlighting}
\end{Shaded}

\begin{longtable}[]{@{}lr@{}}
\caption{RMSE}\tabularnewline
\toprule
Method & RMSE\tabularnewline
\midrule
\endfirsthead
\toprule
Method & RMSE\tabularnewline
\midrule
\endhead
Baseline & 1.0600537\tabularnewline
User \& Movie Effect & 0.8244714\tabularnewline
User \& Movie Effect on validation & 0.8651772\tabularnewline
\bottomrule
\end{longtable}

We can see above that this RMSE is higher than the RMSE on the test set.
This is highly probable, given that this was unseen data. The good thing
is that the difference is just 0.0407059. Now, let's see if
\emph{regularisation} will give us better results. \newpage

\hypertarget{regularisation}{%
\subsection{\texorpdfstring{\textbf{Regularisation:}}{Regularisation:}}\label{regularisation}}

The linear model provides a good estimation for the ratings, but doesn't
consider that many movies have very few number of ratings, and some
users rate very few movies. This means that the sample size is very
small for these movies and these users. Statistically, this leads to
large estimated error.

The estimated value can be improved adding a factor that penalizes small
sample sizes and have have little or no impact otherwise. Thus,
estimated movie and user effects can be calculated with these formulas:

\textbf{Movie Effect on Regularisation:}
\[m_i=\frac{1}{N_i+\lambda}\sum_{i=1}^{N_i}(y_{u,i}-\hat{\mu})\]
\textbf{User Effect on Regularisation:}
\[u_i=\frac{1}{N_u+\lambda}\sum_{i=1}^{N_i}(y_{u,i}-m_i-\hat{\mu})\] For
values of N smaller than or similar to \(\lambda\), \(m_i\) and \(u_i\)
is smaller than the original values, whereas for values of N much larger
than \(\lambda\), \(m_i\) and \(u_i\) change very little.

The regularisation process will evaluate different values for
\(\lambda\), delivering to us the corresponding RMSE.

\hypertarget{rmse_curve_regularisation_test}{%
\subsection{\texorpdfstring{\textbf{RMSE\_Curve\_Regularisation\_Test:}}{RMSE\_Curve\_Regularisation\_Test:}}\label{rmse_curve_regularisation_test}}

\includegraphics{MovieLens_files/figure-latex/RMSE_curve_test-1.pdf}

Optimum \(\lambda\) for Test: 4.6\\
RMSE for Test data: 0.864032 \newpage

\begin{longtable}[]{@{}lr@{}}
\caption{RMSEs}\tabularnewline
\toprule
Method & RMSE\tabularnewline
\midrule
\endfirsthead
\toprule
Method & RMSE\tabularnewline
\midrule
\endhead
Baseline & 1.0600537\tabularnewline
User \& Movie Effect & 0.8244714\tabularnewline
User \& Movie Effect on validation & 0.8651772\tabularnewline
User \& Movie Effect Regularisation & 0.8640320\tabularnewline
\bottomrule
\end{longtable}

The \emph{regularisation} gives us a higher RMSE than the first ``User
\& Movie Effect'' model. This is unexpected but since we have shrunk the
population of the original data set, it is kind of expected.

\hypertarget{regularisation-on-validation-data-set}{%
\subsection{\texorpdfstring{\textbf{Regularisation on \emph{validation}
data
set:}}{Regularisation on validation data set:}}\label{regularisation-on-validation-data-set}}

It is time to see the effect of \emph{regularisation} on the validation
data set.

\begin{Shaded}
\begin{Highlighting}[]
\NormalTok{RMSE_function_val_reg <-}\StringTok{ }\KeywordTok{sapply}\NormalTok{(lambda_values, }\ControlFlowTok{function}\NormalTok{(l)\{}
  
\NormalTok{  mu <-}\StringTok{ }\KeywordTok{mean}\NormalTok{(edx}\OperatorTok{$}\NormalTok{rating)}
  
\NormalTok{  m_i <-}\StringTok{ }\NormalTok{edx }\OperatorTok
\StringTok{    }\KeywordTok{group_by}\NormalTok{(movieId) }\OperatorTok
\StringTok{    }\KeywordTok{summarize}\NormalTok{(}\DataTypeTok{m_i =} \KeywordTok{sum}\NormalTok{(rating }\OperatorTok{-}\StringTok{ }\NormalTok{mu)}\OperatorTok{/}\NormalTok{(}\KeywordTok{n}\NormalTok{()}\OperatorTok{+}\NormalTok{l))}
  
\NormalTok{  u_i <-}\StringTok{ }\NormalTok{edx }\OperatorTok
\StringTok{    }\KeywordTok{left_join}\NormalTok{(m_i, }\DataTypeTok{by=}\StringTok{"movieId"}\NormalTok{) }\OperatorTok
\StringTok{    }\KeywordTok{group_by}\NormalTok{(userId) }\OperatorTok
\StringTok{    }\KeywordTok{summarize}\NormalTok{(}\DataTypeTok{u_i =} \KeywordTok{sum}\NormalTok{(rating }\OperatorTok{-}\StringTok{ }\NormalTok{m_i }\OperatorTok{-}\StringTok{ }\NormalTok{mu)}\OperatorTok{/}\NormalTok{(}\KeywordTok{n}\NormalTok{()}\OperatorTok{+}\NormalTok{l))}
  
\NormalTok{  predicted_val_reg <-}\StringTok{ }\NormalTok{validation }\OperatorTok
\StringTok{    }\KeywordTok{left_join}\NormalTok{(m_i, }\DataTypeTok{by =} \StringTok{"movieId"}\NormalTok{) }\OperatorTok\StringTok{ }
\StringTok{    }\KeywordTok{left_join}\NormalTok{(u_i, }\DataTypeTok{by =} \StringTok{"userId"}\NormalTok{) }\OperatorTok
\StringTok{    }\KeywordTok{mutate}\NormalTok{(}\DataTypeTok{pred =}\NormalTok{ mu }\OperatorTok{+}\StringTok{ }\NormalTok{m_i }\OperatorTok{+}\StringTok{ }\NormalTok{u_i) }\OperatorTok\StringTok{ }\KeywordTok{mutate}\NormalTok{(}\DataTypeTok{pred =} \KeywordTok{ifelse}\NormalTok{(pred }\OperatorTok{<}\StringTok{ }\DecValTok{0}\NormalTok{, }\DecValTok{0}\NormalTok{, }\KeywordTok{ifelse}\NormalTok{(pred }\OperatorTok{>}\StringTok{ }\DecValTok{5}\NormalTok{, }\DecValTok{5}\NormalTok{, pred))) }\OperatorTok\StringTok{ }\NormalTok{.}\OperatorTok{$}\NormalTok{pred}
  
  \KeywordTok{return}\NormalTok{(}\KeywordTok{RMSE}\NormalTok{(predicted_val_reg, validation}\OperatorTok{$}\NormalTok{rating, }\DataTypeTok{na.rm =}\NormalTok{ T))}
\NormalTok{\})}
\end{Highlighting}
\end{Shaded}

\hypertarget{rmse_curve_regularisation_validation}{%
\subsection{\texorpdfstring{\textbf{RMSE\_Curve\_Regularisation\_Validation:}}{RMSE\_Curve\_Regularisation\_Validation:}}\label{rmse_curve_regularisation_validation}}

\includegraphics{MovieLens_files/figure-latex/RMSE_Curve_Val-1.pdf}

Optimum \(\lambda\) for Validation: 4.9

Final RMSE on Regularization: 0.8647188\\
\newpage

\hypertarget{results}{%
\section{\texorpdfstring{\textbf{Results:}}{Results:}}\label{results}}

\begin{longtable}[]{@{}lr@{}}
\caption{RMSEs}\tabularnewline
\toprule
Method & RMSE\tabularnewline
\midrule
\endfirsthead
\toprule
Method & RMSE\tabularnewline
\midrule
\endhead
Baseline & 1.0600537\tabularnewline
User \& Movie Effect & 0.8244714\tabularnewline
User \& Movie Effect on validation & 0.8651772\tabularnewline
User \& Movie Effect Regularisation & 0.8640320\tabularnewline
User \& Movie Effect Reg. on validation & 0.8647188\tabularnewline
\bottomrule
\end{longtable}

We can observe that better RMSE is obtained from the \emph{User \& Movie
Effect} model. However, this RMSE is \emph{only} obtained on the
\emph{test} set. Considering that we must trust more in the performance
of the model when we predict from unseen data, we can say that the RMSE
that results from the \emph{User \& Movie Effect with Regularisation on
validation} (the last line in the table above) is our definitive model.
This RMSE is obtained when \(\lambda\)=4.9 which has achieved
\textbf{RMSE equal to} 0.8647188, successfully passing the target of
0.8649.

\hypertarget{conclusion}{%
\section{\texorpdfstring{\textbf{Conclusion:}}{Conclusion:}}\label{conclusion}}

We started collecting and preparing the dataset for analysis, then we
explored the information seeking for insights that might help during
model build. Next, we created a random model that predicts the rating
based on the probability distribution of each rating. This model gives
the worst result.

We started the linear model with a very simple model which is just the
mean of the observed ratings. From there, we added movie and user
effects, that models the user behavior and movie distribution. With
regularization we added a penalty value for the movies and users with
few number of ratings.

The variables \texttt{userId} and \texttt{movieId} have sufficient
predictive power and could make better recommendations about movie to
specific users of the streaming service. Therefore, the user could
decide to spend more time using the service.

The RMSE equal to 0.8647188 is pretty acceptable considering that we
have few predictors, but both \emph{User} and \emph{Movie} effects are
powerful enough to predict the \texttt{rating} that will be given to a
movie, by a specific user.

\hypertarget{limitations-and-future-work}{%
\section{\texorpdfstring{\textbf{Limitations and Future
Work:}}{Limitations and Future Work:}}\label{limitations-and-future-work}}

The model works only for existing users, movies and rating values, so
the algorithm must run every time a new user or movie is included, or
when the rating changes. This is not an issue for small client base and
a few movies, but may become a concern for large data sets. The model
should consider these changes and update the predictions as information
changes.

There is definitely a time delta effect on the average rating for a
movie, based on either Actors associated to the movie or new movies that
have similar genres and similar story lines that could inspire more
users to rate older movies and hence can bring a significant change to
the predicting model.

Only two predictors are used, the movie and user information, not
considering other features. Modern recommendation system models use many
predictors, such as genres, bookmarks, play lists, history, etc.

From a future perspective we can look at using Matrix Factorization with
other ensemble methods to create a much more powerful predicting
algorithm and develop a penalty based system for the whole algorithm to
improve its performance over time.

\end{document}
